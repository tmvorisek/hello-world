\documentclass{article}
\usepackage[paper=a4paper,margin=0.75in]{geometry}
\usepackage{listings}
\usepackage{color}
\usepackage{graphicx}
\usepackage{amsmath}
\graphicspath{ {images/} }
 
\definecolor{codegreen}{rgb}{0,0.6,0}
\definecolor{codegray}{rgb}{0.5,0.5,0.5}
\definecolor{codepurple}{rgb}{0.58,0,0.82}
\definecolor{backcolour}{rgb}{0.95,0.95,0.92}

\lstdefinestyle{mystyle}{
    backgroundcolor=\color{backcolour},   
    commentstyle=\color{codegreen},
    keywordstyle=\color{blue},
    numberstyle=\tiny\color{codegray},
    stringstyle=\color{codepurple},
    basicstyle=\footnotesize,
    breakatwhitespace=false,         
    breaklines=true,                 
    captionpos=b,                    
    keepspaces=true,                 
    numbers=left,                    
    numbersep=5pt,                  
    showspaces=false,                
    showstringspaces=false,
    showtabs=false,                  
    tabsize=2
}
 
\lstset{style=mystyle}

\begin{document}
  \null\hfill\begin{tabular}[t]{l@{}}
  \textbf{Todd Vorisek} \\
  CS 372 \\
  Homework 1\\
  \textit{Feb 7, 2018} \\
  \end{tabular}\\

  Github username: tmvorisek
  \section*{Steps involved}
    \begin{itemize}
      \item In github, press the plus and select new project. Found at https://github.com/tmvorisek/hello-world
      \item Generate some ssh keys for new github account, so I don't have to keep signing in on pushes/pulls, log them with github.  
      \item Clone the new repo. 
      \item Open sublime text and write new fish file:\\\lstinputlisting[language=bash, caption=Fish file.]{test.fish} 
      \item In a terminal add, commit and push the new file.
      \item In sublime text, make an edit to the new fish file, then add and commit the change.
      \item In github, make a different edit to the same line, then branch the change.
      \item In sublime text run merge, then do the grizzly business.
      \item Back in terminal push the merged file, and call it done.
    \end{itemize}
    Below are 2  screen shots of the tools used and the github network graph:\\
    \includegraphics[width=\textwidth]{images/shot-2018-02-06_23-17-28.jpg}
    \includegraphics[width=\textwidth]{images/shot-2018-02-06_23-19-39.jpg}

  \section*{3 Git Commands}
    \subsection{revert}
      \begin{lstlisting} 
        git revert [options] <commit>\end{lstlisting}
      Get rid of one or more \verb|<commit>|. \verb|revert| effectively generates new commits that undo the changes present in the target commits. \verb|revert| can also be used to alter an existing commits message, which I know I personally will probably find useful. \\
    \subsection{mv}
      \begin{lstlisting} 
        git mv [-v] [-f] [-n] [-k] <source> <destination>\end{lstlisting} 
      Moves a file or folder specified by \verb|<source>| and stages those changes for commit, but the commits must still be performed my the user to add them to the current branch.\\
    \subsection{grep}
      \begin{lstlisting} 
        git grep [options] [-e] <pattern> [<rev>...] [[--] <path>...]\end{lstlisting} 
      Search for \verb|<pattern>| in the tracked files of the current branch's work tree. Specify \verb|-w| or \verb|--word-regexp| to use regular expressions instead of simple pattern matching. Multiple text patterns can be specified by separating them with newline characters. 
\end{document}

